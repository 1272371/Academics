\documentclass[pdf]{prosper}

% Presentation on the make utility and other *n?x goodies
% Created by: Ramprasad Joshi
%             --Adapted from:
%             my preentation RealCombinatorial.tex

\usepackage[toc,highlight,HA]{HA-prosper}

\HAPsetup{%
trans=Dissolve,
tsnav=FullScreen,
nsnav=ShowBookmarks,
lf={\href{mailto:rsj@bits-goa.ac.in}{Ramprasad Joshi}, \today},
rf={Programming Tools on *n?x},
iacolor=gray,
stype=1
}

\title{Programming Tools on *n?x}
\subtitle{make, gdb, shell, etc.}
\author{LUG\\
\institution{BITS, Pilani - K. K. Birla Goa Campus}\\
\institution{\href{http://www.bits-goa.ac.in/CSIS/CSIS.htm}{http://www.bits-goa.ac.in/CSIS/CSIS.htm}}}


\begin{document}

% ==================================================================================
% Slide 1
\maketitle
% ==================================================================================


% ==================================================================================
% Slide 2
\tsectionandpart{Introduction}
% ==================================================================================


% ==================================================================================
% Slide 3
\overlays{4}{
\begin{slide}{Brief}
Programming on a GNU/Linux is fun as well as efficient. The rich repertoire of tools available (as free software) makes it both.
\begin{itemstep}
  \xitem The {\tt gcc} system is actually much more than a C compiler: it is a posse of compilers -- C, C++, Java, Fortran, Common Lisp, and more
  \xitem The {\tt make} utility is a powerful utility not only to compile incrementally (only the refreshed/updated files) but also to rebuild or reprocess any file that has been updated.
  \xitem The debugger {\tt gdb} is a powerful source-level debugger for all the languages that {\tt gcc} compiles, and it is programmable itself! You can even write scripts for automating debugging and test-data collecting tasks in {\tt gdb} non-interactively.
\end{itemstep}
\end{slide}
}
% ==================================================================================


% ==================================================================================
% Slide 4
\overlays{2}{
\begin{slide}{Outline}
\begin{itemstep}
  \xitem We are going to discuss:
  \begin{itemize}
    \xitem How to write programs that need not be tested interactively
    \xitem How to construct test files
    \xitem How to use {\tt make}
    \xitem How to debug using {\tt gdb}
    \xitem Shell basics
  \end{itemize}
\end{itemstep}
\end{slide}
}
% ==================================================================================


% ==================================================================================
% Slide 5
\tsectionandpart{Non-interactive Programming}
% ==================================================================================


% ==================================================================================
% Slide 6
\overlays{6}{
\begin{slide}[trans=Wipe]{Input-output}
\begin{itemstep}
  \xitem {\tt scanf} automatically consumes all and any whitespace, however long
  \xitem Use only the format flags for the variables in {\tt scanf}
  \xitem Such a program can then be run with input redirection
  \xitem Like: ./a.out < inputfile
  \xitem This makes your testing faster too: repeated trials with small changes to the code are now possible easily with a ready test input
  \xitem *USE SHELL HISTORY*
\end{itemstep}
\end{slide}
}
% ==================================================================================


% ==================================================================================
% Slide 7
\overlays{4}{
\begin{slide}{Test files}
\begin{itemstep}
  \xitem Always read input sizes and other parameters that decide how to read the input at the beginning
  \xitem Fix a standard sequence for such parameters. e.g. Array sizes in sorting
  \xitem Never use promts. Instead, prepare test files incrementally with the smallest or shortest input in one file, medium-sized in another, etc.
  \xitem Use a random number generator program for generating test files with varying sizes and combinations (I'll share mine)
\end{itemstep}
\end{slide}
}
% ==================================================================================


% ==================================================================================
% Slide 8
\overlays{4}{
\begin{slide}{Planning the tests}
\begin{itemstep}
  \xitem When writing a complex program, first write a skeleton that pretends to follow the same sequence of major steps except the actual computations
  \xitem Prepare the test files and run the skeletal program on them to see if the input-output part is alright
  \xitem Insert prompts at important points, but bracket them with preprocessor directives
  \xitem And compile with the debug macro defined while testing
\end{itemstep}
\end{slide}
}
% ==================================================================================


% ==================================================================================
% Slide 9
\overlays{5}{
\begin{slide}[trans=Replace]{Using {\tt make}}
See the sample makefile.
\begin{itemstep}
 \xitem CC, LD, RM, CFLAGS, LDFLAGS, MAIN are shell-like variables
 \xitem OBJS is a pattern substitution formula, a special {\tt make} variable
 \xitem Anything that ends with a colon ':' is a target
 \xitem A target is followed by its dependencies on the same line
 \xitem On the following line, the means to achieve the target, using the variables and shell commands, and other targets
\end{itemstep}

\end{slide}
}
% ==================================================================================


% ==================================================================================
% Slide 10
\overlays{2}{
\begin{slide}[trans=Replace,toc=Exploiting Multitasking-1]{}
\begin{itemstep}
  \xitem We may not be, but \texttt{*n?x} systems \emph{are} multitasking, and this is not a marketing gimmick
  \xitem Very rarely these systems hang because of overload. Not much overheating too, unless the PC is bad. I have run laptops on \texttt{GNU/Linux} for weeks at a stretch doing hard numbercrunching unattended, handling even long powerdowns without need to rerun, \emph{unattended}. All you need to do is organize and do some shell-programming.
\end{itemstep}
\end{slide}
}
% Slide 11
\overlays{3}{
\begin{slide}[trans=Replace,toc=Exploiting Multitasking-2]{}
\begin{itemstep}
  \xitem First, open many terminals. One runs a \texttt{vi} clone. The \texttt{C}-program is always open in it, and you save it (\texttt{:w} and \emph{not} \texttt{:w\textbf{q}}) everytime you switch to another window (but \textbf{not} close this one).
  \xitem In another terminal, run \texttt{gcc} or better \texttt{make} everytime you modify and save the program, to test the effect of the change.
  \xitem In yet other terminals, you can keep open (running \texttt{vi} clones) your test input files, modify them time to time, save and test their effects in a similar manner. Of course, if you only change input, you do not need to recompile.
\end{itemstep}
\end{slide}
}
% ==================================================================================


% ==================================================================================
% Slide 12
\tsectionandpart{{Debugging with {\tt gdb}}}
% ==================================================================================


% ==================================================================================
% Slide 13
\overlays{8}{
\begin{slide}{How to compile for debugging}
\begin{itemstep}
  \xitem You need to compile with the \texttt{-g} flag, among others. (i.e. \texttt{\$ gcc -g ...})
  \xitem Then the executable binary generated can be debugged with \texttt{gdb}. Or \texttt{ddd}.
  \xitem For all languages compiled by the \texttt{GCC}, the \textsc{GNU Compiler Collection}, \texttt{gdb} can be used.
  \xitem e.g. C++ (using the \texttt{g++} compiler), Java (using \texttt{gcj}), Fortran (\texttt{gfortran}), etc.
  \xitem Many IDEs, including \texttt{Eclipse}, have in-built source-level visual debuggers, all using \texttt{gdb} at the lowest level.
  \xitem \texttt{gdb} lets you run a program step-by-step
  \xitem -   lets you set breakpoints and run up till any of them is encountered
  \xitem -   lets you finish the current function (all the remaining lines in it as a single step)
\end{itemstep}
\end{slide}
}
% ==================================================================================


% ==================================================================================
% Slide 14
\overlays{4}{
\begin{slide}{More on \texttt{gdb}}
\begin{itemstep}
  \xitem \texttt{gdb} lets you interrupt a program without loss of data
  \xitem -  then examine and possibly modify contents of registers and variables
  \xitem -  and continue running un interrupted again or step by step thenceforth.
  \xitem -  And do many many many more things, including scripting and off-line unattended debugging!
\end{itemstep}
\end{slide}
}
% ==================================================================================


% ==================================================================================
% Slide 15
\overlays{3}{
\begin{slide}{Laborare est Orare}
\begin{itemstep}
  \xitem Seeing is believing ...
  \xitem The taste of the pudding is in eating it ...
  \xitem The test of the program is after gdbing it ...
\end{itemstep}
\end{slide}
}
% ==================================================================================


% ==================================================================================
% Slide 16
\overlays{6}{
\begin{slide}{Concrete Examples}
\begin{itemstep}
  \xitem Look at the program BubbleSort.c bundled with this.
  \xitem Also see the Makefile. It is written to compile any standalone C source file into an executable.
  \xitem So if you have name1.c and name2.c in the current directory, both with a main() in them, then issuing \texttt{make} will make two executables \texttt{xname1} and \texttt{xname2}
  \xitem In BubbleSort, there are ``\texttt{assert}''s to check correctness at various important points
  \xitem Issue \texttt{make debug} and build debug versions of the programs
  \xitem Then use \texttt{gdb} to trace their executions to check how the \texttt{assert}s work.
\end{itemstep}
\end{slide}
}
% ==================================================================================


% ==================================================================================
% Slide 17
\overlays{8}{
\begin{slide}{A Shell Script}
\begin{itemstep}
 \xitem Issue the following commands in a \texttt{bash} shell when the pwd is the directory containing the Makefile and BubbleSort.c bundled here:
 \xitem \texttt{\$ make debug}
 \xitem \texttt{\$ for ((i=1; i<=1000; i++)) ; do}
 \xitem \texttt{\$ ((n=\$RANDOM \% 1000)) ; echo \$n > sortestin}
 \xitem \texttt{\$ for ((i=0; i<\$n; i++)) ; do echo \$RANDOM ; done >> sortestin}
 \xitem \texttt{\$ ./xBubbleSort < sortestin}
 \xitem \texttt{\$ done > sortestout}
 \xitem Wait patiently. Now (after quite some time) you get in \texttt{sortestout} two columns: n and the no. exchanges of BubbleSort on the randomly generated input of size n. To speed up, use just \texttt{make} instead of \texttt{make debug} above.
\end{itemstep}
\end{slide}
}
% ==================================================================================


% ==================================================================================
% Slide 18
\overlays{2}{
\begin{slide}[toc=Finish]{Plotting the Graph}
\begin{itemstep}
 \xitem Now plot the graph by taking the two columns in \texttt{sortestout} as x and y columns
 \xitem Use \texttt{gnuplot} or, if you are weak-hearted, \texttt{LibreOffice Calc}
\end{itemstep}
\end{slide}
}
% ==================================================================================
% Slide 19
\tsectionandpart{Next Part}
% ==================================================================================


% ==================================================================================
% Slide 18
\overlays{6}{
\begin{slide}{}
\begin{itemstep}
  \xitem Stop
  \xitem Watch
  \xitem and
  \xitem Do not go
  \xitem Until you do what this tells you to do
  \begin{itemize}
    \xitem Work
    \xitem in
    \xitem Progress
  \end{itemize}
  \xitem Or regress ? ;-)
\end{itemstep}
\end{slide}
}
% ==================================================================================

\end{document}
